\documentclass{article}

\usepackage[ansinew,latin1]{inputenc}
\usepackage[ngerman]{babel}

\begin{document}

\noindent
Es seien $(x_1|y_1)$ und $(x_2|y_2)$ die Mittelpunkte der B�lle, ${v_{x2}\choose v_{y1}}$ und ${v_{x1}\choose v_{y2}}$ ihre Geschwindigkeitsvektoren, $r$ ihre Radien und $t$ die Zeit, wobei $t=0$ die momentane Situation ist. Gesucht wird $t$, sodass sich die beiden B�lle ber�hren, das hei�t dass ihr Abstand genau $2r$ ist. Folgende Gleichung ergibt sich aus dem Satz des Pythagoras und der Bewegungsgleichung bei konstanter Geschwindigkeit ($x=x_0+vt$):\\
$(x_1+v_{x1}-x_2-v_{x2}^2)^2+(y_1+v_{y1}-y_2-v_{y2})^2=(2r)^2$\\

\noindent
Die Gleichung wird nach $t$ aufgel�st und ergibt dabei ein Polynom vom Grad 2:\\
$(\Delta x+\Delta v_x)^2+(\Delta y+\Delta v_y)^2-4r^2=0$\\
$\Delta x^2+2\Delta x\Delta v_xt+\Delta v_x^2t^2+\Delta y^2+2\Delta y\Delta v_yt+\Delta v_y^2t^2-4r^2=0$\\
$(\Delta v_x^2+\Delta v_y^2)t^2+2(\Delta x\Delta v_x+\Delta y\Delta v_y)t+\Delta x^2+\Delta y^2-4r^2=0$\\

\noindent
$t$ kann nun mithilfe der L�sungsformel bestimmt werden. Dabei ergeben sich bis zu zwei L�sungen, wobei nur die kleinere verwendet wird, da die gr��ere nur korrekt w�re, wenn die B�lle sich nicht absto�en w�rden.\\
$t=\frac{-b-\sqrt{b^2-4ac}}{2a}$\\

\noindent
$a$, $b$ und $c$ lassen sich direkt aus der obigen Gleichung ablesen:\\
$a=\Delta v_x^2+\Delta v_y^2$\\
$b=2(\Delta x\Delta v_x+\Delta y\Delta v_y)$\\
$c=\Delta x^2+\Delta y^2-4r^2$\\

\noindent
Mithilfe der Diskriminante $b^2-4ac$ l�sst sich bestimmen, ob �berhaupt eine Kollision stattfindet. Ist sie nichtnegativ, so kollidieren die B�lle (bei konstanter Geschwindigkeit), ist sie negativ, so kann das Ergebnis verworfen werden.\\

\noindent
Im Spiel bewegen sich die B�lle jedoch mit Reibung. Die Reibung hat jedoch keinen Einfluss auf die Punkte, an denen die B�lle kollidieren (falls sie kollidieren). Um herauszufinden, ob und wann die B�lle mit Reibung kollidieren, wird die Position zum Zeitpunkt $t$ berechnet. Dies ist die Position, an der die B�lle - mit oder ohne Reibung - kollidieren. Um nun die Zeit $t'$ zu berechnen, zu der diese Position mit Reibung erreicht wird, wird die Funktion $adjustCollisionTime$ verwendet, die hier jedoch nicht weiter erl�utert wird.

\end{document}